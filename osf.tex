\documentclass[a4paper]{article}
\usepackage[letterpaper]{geometry}
\usepackage[utf8]{inputenc}
\usepackage[T1]{fontenc}
\pagestyle{empty}
\usepackage[polish]{babel}
\usepackage{enumerate}


\begin{document}
\title{Opis projektu badawczego}
\author{Agnieszka Biernacka\\ Katarzyna Koptyra}
\maketitle

\section{Cel naukowy projektu}
Celem opisywanego projektu jest zbadanie możliwości przesyłania informacji w miejscach, z których korzystają niepełnosprawni i w których takie sygnały mogłyby posłużyć do przesyłania informacji pomocniczych. Prowadzone badania będą opierały się na danych uzyskanych z urządzeń BlueTooth rozmieszczonych w środkach komunikacji miejskiej, urzędach i szpitalach. Do zadań przedstawionego projektu należą analiza zasięgu sygnału oraz szybkości jego odbioru dla urządzeń nadawczych i odbiorczych. Dodatkowo zostaną oszacowane najbardziej optymalne parametry nadajników i odbiorników w wybranych miejscach. Uzyskane wyniki wykorzystane zostaną do realizacji głównego celu, jakim jest zbadanie możliwości przesyłania informacji przydatnych dla osób niepełnosprawnych w wirtualnej sieci.

\section{Znaczenie projektu}
W projekcie została przyjęta następująca metodyka badawcza. Początkowo wyznaczone zostaną optymalne lokalizacje urządzeń do wysyłania i odbioru sygnału w wybranych miejscach, jakimi są tramwaje, autobusy, urzędy i szpitale. Następnie dobrane zostaną parametry nadajników i odbiorników w oparciu o wyznaczone wcześniej lokalizacje. Badaniu będzie podlegał zasięg sygnału w wybranych miejscach, a także szybkość odbioru sygnału. Uzyskane wyniki będą wykorzystane do oszacowania możliwości wykorzystania przesyłanych informacji w celu pomocy osobom niepełnosprawnym, a także opracowanie i wybór najkorzystniejszych metod pomocy.

\section{Koncepcja i plan badań}
Wyznaczenie optymalnych lokalizacji urządzeń BlueTooth w ośrodkach pożytku publicznego.
Badanie zasięgu sygnału w wyznaczonych miejscach.
Badanie szybkości odbioru sygnału.
Opracowanie najbardziej optymalnych parametrów nadajników i odbiorników BlueTooth.

\section{Metodyka badań}
Rezultaty badań mogą mieć znaczący wpływ na podniesienie standardu życia osób starszych oraz niepełnosprawnych. Osoby z różnymi dysfunkcjami organizmu stanowią liczną grupę społeczną, która, biorąc pod uwagę prognozy statystyczne oraz demograficzne, będzie się powiększać. Należy zaznaczyć, że już w wieku produkcyjnym duży odsetek stanowią osoby niesprawne, a zjawisko to nasila się wraz z wiekiem. Fakt ten w połączeniu z postępującym starzeniem się społeczeństwa wpływa na konieczność dostosowania przestrzeni publicznej dla takich osób w celu ułatwienia im codziennego funkcjonowania. Prowadzone badania mogą przyczynić się do opracowania wirtualnej sieci dla osób niepełnosprawnych. Sieć taka umożliwiłaby odbieranie informacji o obecności takich osób i wysyłanie im informacji pomocnicznych, na przykład wiadomości o nadjeżdżającym tramwaju. Wiele miejsc w przestrzeni publicznej nie jest przystosowanych do wjazdu wózków i brak jest możliwości zbudowania podjazdów. Taka sytuacja występuje między innymi w niektórzych urzędach w Krakowie, które położone są w kamienicach. Zastosowanie sieci umożliwiłoby udzielenie w takim wypadku pomocy. Pracownik byłby powiadamiany o obecności osoby niepełnosprawnej np. przy wejściu do budynku i mógłby odpowiednio zareagować. Uzyskane wyniki pomogą oszacować najkorzystniejsze rozmieszczenie i parametry urządzeń. Podnoszenie standardu życia osób niesprawnych przyczynia się do większej aktywności takich ludzi w życiu publicznym i społecznym, co niesie korzyści zarówno dla nich samych, jak i dla państwa, w którym żyją. Z tego względu ważne jest przystosowanie przestrzeni publicznej dla ich wygody i bezpieczeństwa.

\section{Literatura}
\begin{enumerate}[1.]
\item S. S. Chawathe: \textit{Low-latency indoor localization using bluetooth beacons}, Intelligent Transportation Systems, 2009. ITSC '09. 12th International IEEE Conference on.
\item S. S. Chawathe: \textit{Beacon Placement for Indoor Localization using Bluetooth},  Intelligent Transportation Systems, 2008. ITSC 2008. 11th International IEEE Conference on.
\item G. Anastasi, E. Borgia, M. Conti, E. Gregori: \textit{IEEE 802.11 ad hoc networks: performance measurements}, Distributed Computing Systems Workshops, 2003. Proceedings. 23rd International Conference on.
\item I. A. Lami, H. S. Maghdid, T. Kusele: \textit{SILS: A Smart Indoors Localisation Scheme Based on on-the-go Cooperative Smartphones Networking Using On-board Bluetooth, WiFi and GNSS}, ION GNSS+ 2014, At Tampa Convention Center, Tampa, Florida.
\item Ling Chen, Ibrar Hussain, Ri Chen, WeiKei Huang, Gencai Chen: \textit{BlueView: A Perception Assistant System for the Visually Impaired}, In Proceedings of the 2013 ACM conference on Pervasive and ubiquitous computing, At Switzerland.
\item Wen-Yuag Shin, Liang0yu Chen, Kun-CHan Lan: \textit{Estimating Walking Distance with a Smart Phone}, Parallel Architectures, Algorithms and Programming (PAAP), 2012 Fifth International Symposium on.
\item Qihe. Wang: \textit{Scheduling and simulation of large scale wireless personal area networks}, OAI.
\item J. Kim, K. W. Nam, I. G. Jang, H. K. Yang, K. G. Kim, J. M. Hwang: \textit{Nintendo Wii remote controllers for head posture measurement: accuracy, validity, and reliability of the infrared optical head tracker.}, Department of Ophthalmology, Seoul National University College of Medicine, Seoul, Korea.
\item S. Lu, S. Shere, Y, Liu, Y. Liu: \textit{Device discovery and connection establishment approach using Ad-Hoc Wi-Fi for opportunistic networks.}, Ninth Annual IEEE International Conference on Pervasive Computing and Communications, PerCom 2011, 21-25 March 2011, Seattle, WA, USA, Workshop Proceedings.
\end{enumerate}
\end{document}

