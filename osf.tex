\documentclass[a4paper]{article}
\usepackage[letterpaper]{geometry}
\usepackage[utf8]{inputenc}
\usepackage[T1]{fontenc}
\pagestyle{empty}
\usepackage[polish]{babel}
\usepackage{enumerate}


\begin{document}
\title{Badanie właściwości nadajników, odbiorników oraz adapterów Bluetooth w~miejscach wykorzystywanych przez osoby niepełnosprawne \\-- opis projektu badawczego}
\author{Agnieszka Biernacka \and Katarzyna Koptyra}
\maketitle

\section{Cel naukowy projektu}
Celem opisywanego projektu jest zbadanie możliwości przesyłania informacji za pośrednictwem urządzeń Bluetooth, w~szczególności tzw. beaconów, w~miejscach, z~których korzystają osoby niepełnosprawne i~w~których takie sygnały mogłyby posłużyć do przesyłania informacji pomocniczych. 

Beacon to urządzenie emitujące stały, unikalny sygnał korzystając z~energooszczędnej technologii Bluetooth Low Energy. Wykorzystanie BLE sprawia, że urządzenie może wytrzymać na niewielkiej baterii nawet rok, a~nowe beacony są w~stanie czerpać energię z~fal elektromagnetycznych, dzięki czemu nie wymagają wymiany baterii.
Beacon nie łączy się z~internetem ani pozostałymi beaconami w~danym miejscu. Odpowiednie aplikacje na urządzeniach odbierających  pozwalają wykorzystać dane wysyłane przez beacony do dalszych procesów.

Innymi zaletami beaconów, które przemawiają za możliwością ich wykorzystania do budowy wirtualnej sieci dla osób niepełnosprawnych, jest ich niewielki koszt oraz wsparcie przez większość nowoczesnych urządzeń mobilnych. Na chwilę obecną beacony wspierane są w~systemie iOS 7 i~nowszym na urządzeniach Apple (od iPhone 4s) wyposażonych w~Bluetooth 4.0. Beacony wspierają też wszystkie urządzenia z~Androidem w~wersji 4.3 w~górę i~wyposażone w~Bluetooth 4.0.

Prowadzone badania będą opierały się na danych uzyskanych z~urządzeń beacon rozmieszczonych w~środkach komunikacji miejskiej, w~urzędach oraz w~szpitalach. Do zadań przedstawionego projektu należą analiza zasięgu sygnału oraz szybkości jego odbioru dla urządzeń nadawczych i~odbiorczych w~zależności od rodzaju barier danego obiektu. Dodatkowo zostaną oszacowane najbardziej optymalne parametry nadajników i~odbiorników w~wybranych miejscach. Uzyskane wyniki wykorzystane zostaną do realizacji głównego celu, jakim jest zbadanie możliwości przesyłania informacji przydatnych dla osób niepełnosprawnych w~wirtualnej sieci miejskiej oraz określenie optymalnych parametrów urządzeń w~zależności od docelowego miejsca urządzenia.

\section{Znaczenie projektu}
Rezultaty badań mogą mieć znaczący wpływ na podniesienie standardu życia osób starszych oraz niepełnosprawnych. Osoby z~różnymi dysfunkcjami organizmu stanowią liczną grupę społeczną, która -- biorąc pod uwagę prognozy statystyczne oraz demograficzne -- będzie się powiększać. Należy zaznaczyć, że już w~wieku produkcyjnym duży odsetek stanowią osoby niepełnosprawne, a~zjawisko to nasila się wraz z~wiekiem. Fakt ten w~połączeniu z~postępującym starzeniem się społeczeństwa wpływa na konieczność dostosowania przestrzeni publicznej dla takich osób w~celu ułatwienia im codziennego funkcjonowania. 

Prowadzone badania mogą przyczynić się do opracowania wirtualnej sieci dla osób niepełnosprawnych. Sieć taka umożliwiłaby odbieranie informacji o~obecności takich osób i~wysyłanie im informacji pomocniczych, na przykład wiadomości o~numerze linii nadjeżdżającego tramwaju, mapy pomieszczenia itp. 

Wiele miejsc w~przestrzeni publicznej nie jest przystosowanych do wjazdu wózków inwalidzkich, nie posiadają również możliwości zbudowania podjazdów. Taka sytuacja występuje między innymi w~niektórych urzędach w~Krakowie, które zlokalizowane są w~kamienicach. Stare budownictwo nie uwzględniało obecnych standardów, a~położenie budynków blisko ulicy uniemożliwia rozbudowę wejść o~podjazdy. Zastosowanie wirtualnej sieci umożliwiłoby udzielenie w~takim wypadku pomocy. Pracownik byłby powiadamiany o~obecności osoby niepełnosprawnej np. przy wejściu do budynku i~mógłby odpowiednio zareagować.

Wyniki uzyskane podczas przeprowadzonych badań pomogą oszacować możliwości wykorzystania urządzeń beacon i~innych urządzeń wyposażonych w~Bluetooth do zbudowania wirtualnej sieci miejskiej. Zostaną również wyznaczone najkorzystniejsze rozmieszczenia i~parametry urządzeń.

Podnoszenie standardu życia osób niesprawnych przyczynia się do większej aktywności takich ludzi w~życiu publicznym i~społecznym, co niesie korzyści zarówno dla nich samych, jak i~dla państwa, w~którym żyją. Z~tego względu ważne jest przystosowanie przestrzeni publicznej dla wygody i~bezpieczeństwa tej grupy społecznej.

\section{Koncepcja i~plan badań}
W pracach wstępnych zebrane zostały informacje o~rodzajach urządzeń do rozsyłania i~odbierania informacji poprzez urządzenia Bluetooth, w~tym o~urządzeniach mobilnych oraz beaconach. Najważniejszymi informacjami są tutaj: szybkości przesyłu danych, maksymalny zasięg określony przez producenta, popularność na obecnym rynku, prognozy rozwoju oprogramowania pod wyznaczony kierunek. Określony został również wstępny obszar publiczny, w~którym przeprowadzone badania mogłyby posłużyć jako teoretyczna podstawa do implementacji wirtualnych sieci dla osób niepełnosprawnych.

W ramach prac projektowych planowane są trzy etapy, które będą następowały po sobie w~sposób chronologiczny, tworząc pojedynczą ścieżkę będącą jednocześnie ścieżką krytyczną.

W pierwszym etapie wyznaczone zostaną optymalne lokalizacje urządzeń do wysyłania i~odbioru sygnału w~wybranych ośrodkach publicznych. Do miejsc tych należeć będą: tramwaje, autobusy, urzędy, szpitale. 

W drugim etapie dobrane zostaną parametry nadajników i~odbiorników w~oparciu o~wyznaczone wcześniej lokalizacje. Badaniu będzie podlegał zasięg sygnału w~wybranych miejscach, a~także szybkość odbioru sygnału. 

Uzyskane w~poprzednich krokach wyniki zostaną wykorzystane w ostatnim etapie do~oszacowania możliwości wykorzystania przesyłanych informacji w~celu pomocy osobom niepełnosprawnym, a~także w~celu opracowania i~wyboru najkorzystniejszych metod pomocy.

\section{Metodyka badań}
Uczestnicy projektu będą wykorzystywać metody i~techniki znane z~wieloletniego doświadczenia opiekuna naukowego. Zostaną wykorzystane takie narzędzia jak:
\begin{itemize}
\item beacony,
\item smartphony wyposażone w~Bluetooth 4.0,
\item urządzenia typu smartwatch,
\item serwer dla klientów urządzeń mobilnych.
\end{itemize} 

Badania będą prowadzone w~ośrodkach publicznych, po uprzednim porozumieniu z~zarządcami poszczególnych ośrodków.

\section{Literatura}
\begin{enumerate}[1.]
\item S. S. Chawathe: \textit{Low-latency indoor localization using bluetooth beacons}, Intelligent Transportation Systems, 2009. ITSC '09. 12th International IEEE Conference on.
\item S. S. Chawathe: \textit{Beacon Placement for Indoor Localization using Bluetooth},  Intelligent Transportation Systems, 2008. ITSC 2008. 11th International IEEE Conference on.
\item G. Anastasi, E. Borgia, M. Conti, E. Gregori: \textit{IEEE 802.11 ad hoc networks: performance measurements}, Distributed Computing Systems Workshops, 2003. Proceedings. 23rd International Conference on.
\item I. A. Lami, H. S. Maghdid, T. Kusele: \textit{SILS: A Smart Indoors Localisation Scheme Based on on-the-go Cooperative Smartphones Networking Using On-board Bluetooth, WiFi and GNSS}, ION GNSS+ 2014, At Tampa Convention Center, Tampa, Florida.
\item Ling Chen, Ibrar Hussain, Ri Chen, WeiKei Huang, Gencai Chen: \textit{BlueView: A Perception Assistant System for the Visually Impaired}, In Proceedings of the 2013 ACM conference on Pervasive and ubiquitous computing, At Switzerland.
\item Wen-Yuag Shin, Liang0yu Chen, Kun-CHan Lan: \textit{Estimating Walking Distance with a Smart Phone}, Parallel Architectures, Algorithms and Programming (PAAP), 2012 Fifth International Symposium on.
\item Qihe. Wang: \textit{Scheduling and simulation of large scale wireless personal area networks}, OAI.
\item J. Kim, K. W. Nam, I. G. Jang, H. K. Yang, K. G. Kim, J. M. Hwang: \textit{Nintendo Wii remote controllers for head posture measurement: accuracy, validity, and reliability of the infrared optical head tracker.}, Department of Ophthalmology, Seoul National University College of Medicine, Seoul, Korea.
\item S. Lu, S. Shere, Y, Liu, Y. Liu: \textit{Device discovery and connection establishment approach using Ad-Hoc Wi-Fi for opportunistic networks.}, Ninth Annual IEEE International Conference on Pervasive Computing and Communications, PerCom 2011, 21--25 March 2011, Seattle, WA, USA, Workshop Proceedings.
\end{enumerate}
\end{document}

