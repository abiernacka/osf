\documentclass[a4paper]{article}
\usepackage[letterpaper]{geometry}
\usepackage[utf8]{inputenc}
\usepackage[T1]{fontenc}
\pagestyle{empty}
\usepackage[polish]{babel}


\begin{document}
\title{Opis projektu badawczego}
\author{Agnieszka Biernacka\\ Katarzyna Koptyra}
\maketitle

Celem opisywanego projektu jest zbadanie możliwości przesyłania informacji w miejscach, z których korzystają niepełnosprawni i w których takie sygnały mogłyby posłużyć do przesyłania informacji pomocniczych. Prowadzone badania będą opierały się na danych uzyskanych z urządzeń BlueTooth rozmieszczonych w środkach komunikacji miejskiej, urzędach i szpitalach. Do zadań przedstawionego projektu należą analiza zasięgu sygnału oraz szybkości jego odbioru dla urządzeń nadawczych i odbiorczych. Dodatkowo zostaną oszacowane najbardziej optymalne parametry nadajników i odbiorników w wybranych miejscach. Uzyskane wyniki wykorzystane zostaną do realizacji głównego celu, jakim jest zbadanie możliwości przesyłania informacji przydatnych dla osób niepełnosprawnych w wirtualnej sieci.

W projekcie została przyjęta następująca metodyka badawcza. Początkowo wyznaczone zostaną optymalne lokalizacje urządzeń do wysyłania i odbioru sygnału w wybranych miejscach, jakimi są tramwaje, autobusy, urzędy i szpitale. Następnie dobrane zostaną parametry nadajników i odbiorników w oparciu o wyznaczone wcześniej lokalizacje. Badaniu będzie podlegał zasięg sygnału w wybranych miejscach, a także szybkość odbioru sygnału. Uzyskane wyniki będą wykorzystane do oszacowania możliwości wykorzystania przesyłanych informacji w celu pomocy osobom niepełnosprawnym, a także opracowanie i wybór najkorzystniejszych metod pomocy.

Rezultaty badań mogą mieć znaczący wpływ na podniesienie standardu życia osób starszych oraz niepełnosprawnych. Osoby z różnymi dysfunkcjami organizmu stanowią liczną grupę społeczną, która, biorąc pod uwagę prognozy statystyczne oraz demograficzne, będzie się powiększać. Należy zaznaczyć, że już w wieku produkcyjnym duży odsetek stanowią osoby niesprawne, a zjawisko to nasila się wraz z wiekiem. Fakt ten w połączeniu z postępującym starzeniem się społeczeństwa wpływa na konieczność dostosowania przestrzeni publicznej dla takich osób w celu ułatwienia im codziennego funkcjonowania. Prowadzone badania mogą przyczynić się do opracowania wirtualnej sieci dla osób niepełnosprawnych. Sieć taka umożliwiłaby odbieranie informacji o obecności takich osób i wysyłanie im informacji pomocnicznych, na przykład wiadomości o nadjeżdżającym tramwaju. Wiele miejsc w przestrzeni publicznej nie jest przystosowanych do wjazdu wózków i brak jest możliwości zbudowania podjazdów. Taka sytuacja występuje między innymi w niektórzych urzędach w Krakowie, które położone są w kamienicach. Zastosowanie sieci umożliwiłoby udzielenie w takim wypadku pomocy. Pracownik byłby powiadamiany o obecności osoby niepełnosprawnej np. przy wejściu do budynku i mógłby odpowiednio zareagować. Uzyskane wyniki pomogą oszacować najkorzystniejsze rozmieszczenie i parametry urządzeń. Podnoszenie standardu życia osób niesprawnych przyczynia się do większej aktywności takich ludzi w życiu publicznym i społecznym, co niesie korzyści zarówno dla nich samych, jak i dla państwa, w którym żyją. Z tego względu ważne jest przystosowanie przestrzeni publicznej dla ich wygody i bezpieczeństwa.

\end{document}

